\subsection*{2.1 Discuta el impacto de las interfaces de usuario sobre confiable}
\textbf{Hermes Robles:}

\begin{quote}
 Una aplicac\'on que es \textit{confiable} es aquella
en la cual un usuario puede depender de ella para
realizar su trabajo. Esta puede contener errores pero no afecta su uso.
\end{quote}

\textbf{Efra\'in Tovar:}

\begin{quote}
 Informalmente el software es confiable si el usuario puede tenerle confianza.
Formalmente la confiabilidad se define en t\'erminos del comportamiento estad\'istico:
la probabilidad de que el software opere como es esperado en un intervalo
de tiempo especificado. Contrariamente a la correctitud que es una cualidad
absoluta, la confiabilidad es relativa. Cualquier desviaci\'on de los requerimientos
hace que el sistema sea incorrecto, por otro lado, si la consecuencia de un error
en el software no es seria, el software incorrectp a\'un puede ser confiable.
\end{quote}

\textbf{Karla Ponce:}

\begin{quote}
 El t\'ermino \textit{confiable} hace referencia al software cuyo
comportamiento entra dentro de los "\textit[Bugs conocidos]",
esto implica que si bien tiene fallos, no afectan su uso.
Haciendo que exista esta dependencia al software de parte del usuario.
\end{quote}

